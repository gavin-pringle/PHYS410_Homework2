\documentclass[10pt]{article}

\usepackage[utf8]{inputenc}
\usepackage{latexsym,amsfonts,amssymb,amsthm,amsmath}
\setlength{\parindent}{0in}
\setlength{\parskip}{\baselineskip}
\setlength{\oddsidemargin}{0in}
\setlength{\textwidth}{6.5in}
\setlength{\textheight}{8.8in}
\setlength{\topmargin}{0in}
\setlength{\headheight}{18pt}

\usepackage[a4paper,margin=1in,footskip=0.25in]{geometry}

\usepackage{listings}
\usepackage{color} %red, green, blue, yellow, cyan, magenta, black, white
\definecolor{mygreen}{RGB}{28,172,0} % color values Red, Green, Blue
\definecolor{mylilas}{RGB}{170,55,241}

\usepackage{graphicx}
\graphicspath{{../output/}}

\def\code#1{\texttt{#1}} % Monospacing shortcut: Use \code{}

\usepackage[colorlinks=true, urlcolor=blue, linkcolor=blue]{hyperref}

\usepackage{verbatim} % For including raw text files in the pdf
\usepackage{float} % For keeping figures in the section where they were called 

\title{PHYS 410 Homework 2}
\author{Gavin Pringle, 56401938}

%%%%%%%%%%%%%%%%%%%%%%%%%%%%%%%%%%%%%%%%%%%%%%%%%%%%%%%%%%%%%%%%%%%%%%%%%%%%%%%%%%%%%%%%%%%%%%%%%%%%%%%
% Start of document
%%%%%%%%%%%%%%%%%%%%%%%%%%%%%%%%%%%%%%%%%%%%%%%%%%%%%%%%%%%%%%%%%%%%%%%%%%%%%%%%%%%%%%%%%%%%%%%%%%%%%%%
\begin{document}

\maketitle

\lstset{language=Matlab,%
    %basicstyle=\color{red},
    breaklines=true,%
    morekeywords={matlab2tikz},
    keywordstyle=\color{blue},%
    morekeywords=[2]{1}, keywordstyle=[2]{\color{black}},
    identifierstyle=\color{black},%
    stringstyle=\color{mylilas},
    commentstyle=\color{mygreen},%
    showstringspaces=false,%without this there will be a symbol in the places where there is a space
    numbers=left,%
    numberstyle={\tiny \color{black}},% size of the numbers
    numbersep=9pt, % this defines how far the numbers are from the text
    emph=[1]{for,end,break},emphstyle=[1]\color{red}, %some words to emphasise
    %emph=[2]{word1,word2}, emphstyle=[2]{style},    
}

%%%%%%%%%%%%%%%%%%%%%%%%%%%%%%%%%%%%%%%%%%%%%%%%%%%%%%%%%%%%%%%%%%%%%%%%%%%%%%%%%%%%%%%%%%%%%%%%%%%%%%%
% Introduction
%%%%%%%%%%%%%%%%%%%%%%%%%%%%%%%%%%%%%%%%%%%%%%%%%%%%%%%%%%%%%%%%%%%%%%%%%%%%%%%%%%%%%%%%%%%%%%%%%%%%%%%
\subsection*{Introduction}

In this homework assignment, the fourth-order Runge-Kutta method for computing numerical solutions of 
ODEs is explored. This is done in stages, culminating in creating a MATLAB function that numerically
integrates an ODE using an algorithm that automatically varies the step size of the integrator in order
to achieve a relative error tolerance. 

First, a function \code{rk4step} is written that computes a single fourth-order Runge-Kutta step for a 
system of coupled first-order ODEs, returning the approximate values of the dependent variables after
a defined time step. The function \code{rk4step} is then used in the function \code{rk4} which computes 
the solution of an initial value problem over a range of values for the independent variable, done by 
taking multiple fourth-order Runge-Kutta steps in a loop. Lastly, the function \code{rk4ad} is written 
which finds the numerical solution of an initial value problem by comparing the results of fourth-order 
Runge-Kutta steps of different sizes and then varying the step size as until the error in the 
approximation is below a specified relative tolerance.

%%%%%%%%%%%%%%%%%%%%%%%%%%%%%%%%%%%%%%%%%%%%%%%%%%%%%%%%%%%%%%%%%%%%%%%%%%%%%%%%%%%%%%%%%%%%%%%%%%%%%%%
% Review of Theory
%%%%%%%%%%%%%%%%%%%%%%%%%%%%%%%%%%%%%%%%%%%%%%%%%%%%%%%%%%%%%%%%%%%%%%%%%%%%%%%%%%%%%%%%%%%%%%%%%%%%%%%
\subsection*{Review of Theory}

\subsubsection*{Casting systems of ODEs in first-order form}

In order to solve complicated ODEs numerically, it is useful to first cast them in a canonical form that 
is easier for a computer program to understand. Any ODE defining the function $y(t)$ that is of the form 
$$f(t, y, y', y'', y^{(3)}, \ldots,  y^{(N)}) = 0$$
can be rewritten as a system of $N$ coupled first-order ODEs for the functions 
$y_i(t), \quad i = 1, 2, 3, \ldots, N$:
\begin{equation}\label{sys_DE}
y_i'(t) \equiv \frac{dy_i}{dt}(t) = f_i(t, y_1, y_2, y_3, \ldots, y_N)
\end{equation}
where $f_i$ are known functions of $t$ and $y_i$. This is equivalent to
\begin{equation}\label{sys_DE_vec}
\mathbf{y'} = \mathbf{f}(t, \mathbf{y}) \quad \textrm{where} \quad \mathbf{y} \equiv (y_1, y_2, y_3, 
\ldots, y_N)
\end{equation}

For example, the function $y^{(4)}(t) = f(t)$ can be written as 
$$y_3' = f, \quad y' = y_1, \quad y_1' = y2, \quad y_2' = y_3$$

\subsubsection*{The fourth-order Runge-Kutta step}

The fourth-order Runge-Kutta step for numerically solving a system of $N$ coupled first-order ODEs is 
defined as:
\begin{equation}\label{rk_step}
y_i(t_0 + h) = y_i(t_0) + \frac{h}{6} (f_{0,i} + 2f_{1,i} + 2f_{2,i} + f_{3,i})
\end{equation}
with the terms $f_{0,i} \;f_{1,i} \;f_{2,i} \;f_{3,i}$ given by 
\begin{align}\label{rk_step_f_defs}
f_{0,i} &= f_i(t_0, \;y_{0,i}) \\
f_{1,i} &= f_i\left(t_0 + \frac{h}{2}, \;y_{0,i} + \frac{h}{2} f_{0,i}\right) \\
f_{2,i} &= f_i\left(t_0 + \frac{h}{2}, \;y_{0,i} + \frac{h}{2} f_{1,i}\right) \\
f_{3,i} &= f_i\left(t_0 + h, \;y_{0,i} + h f_{2,i}\right)
\end{align}
where each $f_i$ on the right-hand sides of the above equations are defined in (\ref{sys_DE}) and 
h is the step size. This can be understood as a weighted sum of four numerical approximations to the 
solution of the ODE. A single fourth-order Runge-Kutta step is accurate to $O(\Delta t^5)$.

%%%%%%%%%%%%%%%%%%%%%%%%%%%%%%%%%%%%%%%%%%%%%%%%%%%%%%%%%%%%%%%%%%%%%%%%%%%%%%%%%%%%%%%%%%%%%%%%%%%%%%%
% Numerical approach
%%%%%%%%%%%%%%%%%%%%%%%%%%%%%%%%%%%%%%%%%%%%%%%%%%%%%%%%%%%%%%%%%%%%%%%%%%%%%%%%%%%%%%%%%%%%%%%%%%%%%%%
\subsection*{Numerical Approach}

\subsubsection*{Consecutive fourth-order Runge-Kutta steps}

In order to compute the numerical solution to an ODE (or in canonical form, a system of first-order 
ODEs), multiple consecutive fourth-order Runge-Kutta steps must be taken. The MATLAB implementation of 
a single fourth-order Runge-Kutta step is shown in Appendix A as \code{rk4step.m}, while Appendix C 
shows the MATLAB implementation of a complete fourth-order Runge-Kutta ODE integrator as \code{rk4.m}. 
In \code{rk4.m}, equation (\ref{rk_step}) is repeatedly computed using the previous output of equation 
(\ref{rk_step}) as an input. The step size \code{dt} is given by the difference at the current time-step 
between independent variable values at which the solution of the ODE is to be computed at, passed as 
\code{tspan}.

It is important to note that while a single fourth-order Runge-Kutta step is accurate to $O(\Delta t^5)$,
the full numerical solution to and ODE given by \code{rk4.m} is accurate to $O(\Delta t^4)$, since
error accumulates linearly throughout the integration. 

\subsubsection*{Adaptive step sizing}

The Runge-Kutta integrator can be made more accurate by varying the step size \code{dt} at each step 
depending on an estimation of the error accumulated in that step. To show how the per-step error can be
estimated using time-steps of multiple lengths, consider (for each dependent variable y in the system of 
ODEs) a "coarse" step of length $\Delta t$ (producing the output $y_C$) and two "fine" steps of length 
$\Delta t / 2$ (producing the output $y_F$).
\begin{align*}\label{y_F_and_y_C}
y_C(t_0 + \Delta t) &\approx y_{\text{exact}}(t_0 + \Delta t) + k(t_0) \Delta t^5 \\
y_F(t_0 + \Delta t) &\approx y_{\text{exact}}(t_0 + \Delta t) + k(t_0) \left( \frac{\Delta t}{2} 
\right)^5 + k\left(t_0 + \frac{\Delta t}{2}\right) \left( \frac{\Delta t}{2} \right)^5 \\
&\approx y_{\text{exact}}(t_0 + \Delta t) + 2 k(t_0) \left( \frac{\Delta t}{2} \right)^5
\end{align*}
In the above equations, $k(t)$ is some function of time. Subtracting $y_F$ and $y_C$ at the advanced time,
we get
\begin{equation}\label{error_estimation}
y_C(t_0 + \Delta t) - y_F(t_0 + \Delta t) \approx \frac{15}{16}k(t_0) \Delta t^5 \approx \frac{15}{16}e_C
\end{equation}
which yields an estimate for the local solution error $e_C$.

This error estimation is implemented in the MATLAB function \code{rk4ad.m} (Appendix F), which functions 
as an adaptive step size fourth-order Runge-Kutta ODE integrator. In \code{rk4ad.m}, similar to 
\code{rk4.m}, the function is written to compute the values of the ODE defined by the argument \code{fcn}
at the times defined in \code{tspan}, with the initial values of the dependent variables defined by 
\code{y0}. However, the additional argument \code{reltol} is also passed which 

% Describe my rk4ad algorithm 

%%%%%%%%%%%%%%%%%%%%%%%%%%%%%%%%%%%%%%%%%%%%%%%%%%%%%%%%%%%%%%%%%%%%%%%%%%%%%%%%%%%%%%%%%%%%%%%%%%%%%%%
% Implementation
%%%%%%%%%%%%%%%%%%%%%%%%%%%%%%%%%%%%%%%%%%%%%%%%%%%%%%%%%%%%%%%%%%%%%%%%%%%%%%%%%%%%%%%%%%%%%%%%%%%%%%%
\subsection*{Implementation}

\subsubsection*{ODE test functions} % sho and vdp
% In order to test ... 

% define ode equations (as in comment block)

%%%%%%%%%%%%%%%%%%%%%%%%%%%%%%%%%%%%%%%%%%%%%%%%%%%%%%%%%%%%%%%%%%%%%%%%%%%%%%%%%%%%%%%%%%%%%%%%%%%%%%%
% Results
%%%%%%%%%%%%%%%%%%%%%%%%%%%%%%%%%%%%%%%%%%%%%%%%%%%%%%%%%%%%%%%%%%%%%%%%%%%%%%%%%%%%%%%%%%%%%%%%%%%%%%%
\subsection*{Results}

\subsubsection*{rk4step.m Output}

\subsubsection*{rk4.m Output}

\subsubsection*{rk4ad.m Output}


%%%%%%%%%%%%%%%%%%%%%%%%%%%%%%%%%%%%%%%%%%%%%%%%%%%%%%%%%%%%%%%%%%%%%%%%%%%%%%%%%%%%%%%%%%%%%%%%%%%%%%%
% Conclusions
%%%%%%%%%%%%%%%%%%%%%%%%%%%%%%%%%%%%%%%%%%%%%%%%%%%%%%%%%%%%%%%%%%%%%%%%%%%%%%%%%%%%%%%%%%%%%%%%%%%%%%%
\subsection*{Conclusions}


\pagebreak

%%%%%%%%%%%%%%%%%%%%%%%%%%%%%%%%%%%%%%%%%%%%%%%%%%%%%%%%%%%%%%%%%%%%%%%%%%%%%%%%%%%%%%%%%%%%%%%%%%%%%%%
% Appendices
%%%%%%%%%%%%%%%%%%%%%%%%%%%%%%%%%%%%%%%%%%%%%%%%%%%%%%%%%%%%%%%%%%%%%%%%%%%%%%%%%%%%%%%%%%%%%%%%%%%%%%%

\subsection*{Appendix A - rk4step.m Code}
\lstinputlisting{../src/rk4step.m}
\pagebreak

\subsection*{Appendix B - trk4step.m Code}
\lstinputlisting{../src/trk4step.m}
\pagebreak

\subsection*{Appendix C - rk4.m Code}
\lstinputlisting{../src/rk4.m}
\pagebreak

\subsection*{Appendix D - trk4\_sho.m Code}
\lstinputlisting{../src/trk4_sho.m}
\pagebreak

\subsection*{Appendix E - trk4\_vdp.m Code}
\lstinputlisting{../src/trk4_vdp.m}
\pagebreak

\subsection*{Appendix F - rk4ad.m Code}
\lstinputlisting{../src/rk4ad.m}
\pagebreak

\subsection*{Appendix G - trk4ad\_sho.m Code}
\lstinputlisting{../src/trk4ad\_sho.m}
\pagebreak

\subsection*{Appendix H - trk4ad\_vdp.m Code}
\lstinputlisting{../src/trk4ad\_vdp.m}
\pagebreak

\end{document}